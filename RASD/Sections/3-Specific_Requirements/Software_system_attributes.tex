\subsection{Reliability}
The system needs to always be online in order to correctly store and manage violation reports and to be robust in case of failures. Moreover a cloud based database can be used in order to guarantee data redundancy and fast access. 
\subsection{Availability}
As previously stated, the system must always guarantee an operating service of violation reports. These requirements can be guaranteed through a virtual replication of the backend system: a possible solution could be implemented through Docker containers managed by a container orchestrion such as Kubernetes or DockerCompose. The container architecture automatically restarts containers in case of failures (fault tolerant) guaranteeing the system functioning in every situation.

\subsection{Security}
As the data transferred and stored in the SafeStreets system deals with
private citizen's data, it must be guaranteed an high level of security.
In order to meet the GDPR conditions, all data has to be correctly 
encrypted through several security methods before being sent and stored.
\subsection{Maintainability}
The software maintainability is one of the most critical aspect in the
development of this software. In order to allow cheap and easy fix, SafeStreets
is developed by using several design patterns that make the software
flexible and scalable. Such patterns can be recognized both in the previous
UML class diagram and in further documents.
\subsection{Portability}
The SafeStreets mobile application is intended to work on every mobile device, thus the technologies used for its development have to be cross-platform (such as Flutter or any other language which can generate applications for IOS and Android systems).
