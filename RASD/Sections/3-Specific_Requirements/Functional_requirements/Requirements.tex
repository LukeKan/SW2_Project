\begin{itemize}
  \item \textbf{{[G1]}: Allow the user to send a violation report consisting in a picture and some metadata;}
    \begin{itemize}
      \item {[R1]}: The mobile application must be able to gather all the required data such as: location, date and time from the user's device;
      \item {[R2]}: The mobile application has to allow the user to choose between different categories of violations;
      \item {[R3]}: The mobile application has to allow the user to optionally add information about the event and the car plate number;
      \item {[R4]}: The mobile application has to allow the user to retake the picture if desired;
      \item {[R5]}: The mobile application must be able to interact with the backend system to send/retrieve information;
      \item {[R6]}: The backend system must be able to send the violation image to the plate recognition service in order to retrieve the car plate number from the image;
      \item {[R7]}: Once the car plate number has been retrieved the system must save the report information in the database;
      \item {[D1]}: A picture taken with a smartphone is performed with a quality sufficient for the image recognition service to transcribe it;
      \item {[D2]}: The reported picture contains only the license plate of the car that committed the violation and not others;
      \item {[D3]}: The GPS information collected from the smartphone of a user has an accuracy of less than 5 meters;
      \item {[D4]}: The timestamp collected from the smartphone is synchronized with the CET standard;
      \item {[D5]}: The users can only report violations occurred in Europe;
      \item {[D6]}: Reports are only sent through a secure connection channel;
      \item {[D7]}: The users reports all the violation that they detect;
      \item {[D8]}: The users send report containing correct information about the detected violation;
    \end{itemize}
  \item \textbf{{[G2]}: Allow the user to watch the history of his reports and their status;}
    \begin{itemize}
      \item {[R5]}: The mobile application must be able to interact with the backend system to send/retrieve information;
      \item {[R8]}: The backend system must retrieve the latest reports submitted by the user from the database. Once the data is gathered, the backend system must send it to the mobile application that will display them;
      \item {[R9]}: The user must be able to consult the status associated with each report;
      \item {[D6]}: Reports are only sent through a secure connection channel;
    \end{itemize}
  \item \textbf{{[G3]}: Allow the user to watch on map areas and streets with an high frequency of violations;}
    \begin{itemize}
      \item {[R5]}: The mobile application must be able to interact with the backend system to send/retrieve information;
      \item {[R10]}: The application has to allow the user to insert an address near which to search the MDS. If no address is provided the application will instead search the MDS near the current position of the user (collected from the GPS);
      \item {[R11]}: The backend system must calculate the MDS close to the given location;
      \item {[R12]}: The mobile application must enlighten on the map the MDS;
    \end{itemize}
  \item \textbf{{[G4]}: Allow the local system administrator of the police station to create accounts for the police technicians;}
    \begin{itemize}
      \item {[R13]}: The Web application must be able to interact with the backend system to send/retrieve information;
      \item {[R14]}: The Web application must provide a special interface for the LSA in order to perform special task reserved for his role;
      \item {[R15]}: The Web application must allow the LSA to securely create account reserved for PT;
      \item {[R16]}: The Web application must allow the LSA to associate the PT's badge number with his related account;
      \item {[R17]}: The system automatically generates a temporary password associated to the new account. Upon the first login, the application asks the PT to change the password;
      \item {[R18]}: The system correctly registers the new accounts and allows access to the Web application to the PT registered;  
      \item {[D14]}: The police corporates can manage only the police technicians of their jurisdiction.     
    \end{itemize}
  \item \textbf{{[G5]}: Allow the authorities to visualize, schedule and change the status of reports submitted by the users;}
    \begin{itemize}
      \item {[R13]}: The Web application must be able to interact with the backend system to send/retrieve information;
      \item {[R19]}: The Web application allows the visualization of reports to both the LSA and the PTs via a dedicated section;
      \item {[R20]}: The Web application allows the LSA to schedule reports to PT by associating their account to the report;
      \item {[R21]}: The Web application allows PTs to visualize their scheduled reports;
      \item {[R22]}: The Web application allows both the LSA and the PTs to change the status of reports (i.e. from "PENDING" to "SOLVED");
      \item {[D13]}: The police corporates and technicians can manage only the violation reports that belong to their jurisdiction;
      \item {[D15]}: The police technicians can only take care of the violations which are assigned to them by the police corporate;
    \end{itemize}
  \item \textbf{{[G6]}: Allow the authorities to mine data to make analysis and retrieve statistics;}
    \begin{itemize}
      \item {[R13]}: The Web application must be able to interact with the backend system to send/retrieve information;
      \item {[R23]}: The Web application provides a section that can be used by both LSA and PTs to mine the data.
      \item {[R24]}: The system must be able to collect the filter criteria inserted by the authorities and then compose a query that will be executed to retrieve the matching data;
      \item {[R25]}: The Web application provides a section that can be used by both LSA and PTs to visualize statistics/metrics about the data;
      \item {[R26]}: The system must offer some functionalities to calculate the most useful statistics/metrics related to the data;
    \end{itemize}
  \item \textbf{{[G7]}: If the municipality offers a service that provides data about car accidents then the system must be able to cross this information with its owns data in order to identify possible unsafe areas;}
    \begin{itemize}
      \item {[R13]}: The Web application must be able to interact with the backend system to send/retrieve information;
      \item {[R27]}: The Web application provides a section from which the authorities can ask for suggestions;
      \item {[R28]}: The backend system must be able to communicate and retrieve data from the municipality service at any given time;
      \item {[R29]}: The system has to merge and find correlations between the violations and the car accidents;
      \item {[R30]}: The system has to identify potentially unsafe areas and then estimate possible interventions based on the correlations that have been found;
      \item {[R31]}: Once the computation is over the Web application must display the suggested interventions on the interface along with the location;
      \item {[D9]}: The municipality system stores correctly all the information concerning car accidents;
      \item {[D10]}: The municipality system allows SafeStreet to retrieve information about the car accident among a certain area;
      \item {[D11]}: The municipality system sends data accordingly to a common standard which is comprehensible by the SafeStreet system. 
    \end{itemize}
\end{itemize}
