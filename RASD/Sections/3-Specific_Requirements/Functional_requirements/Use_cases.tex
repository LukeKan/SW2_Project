% Sign Up -------------------
\begin{table}[H]
  \centering
  % \renewcommand{\arraystretch}{0.8}
  \begin{tabularx}{\textwidth}{ |l|X| }
    \hline
    Name & Sign Up\\
    \hline
    Actors & Visitor\\
    \hline
    Entry Conditions & The visitor has opened the application on his device\\
    \hline
    Event flow & \begin{enumerate}[noitemsep]
                    \item The Visitor chooses the Sign Up option;
                    \item The Visitor fills the mandatory fields regarding his personal data (i.e. email, password);
                    \item The Visitor fills the optional fields (i.e. Name, Surname, age)
                    \item The Visitor confirms the registration;
                    \item The system saves the information and creates the User account.
                \end{enumerate}\\
    \hline
    Exit conditions & The User is registered.\\
    \hline
    Exceptions & \begin{enumerate}[noitemsep]
                  \item The user was already registered. In this case the application warns the user that there is another account associated to the specified email;
                  \item The user does not fills all the mandatory fields. In that case the application warns the user by highlighting the empty fields.
                \end{enumerate}\\
    \hline
  \end{tabularx}
\end{table}
% Log in --------------------
\begin{table}[H]
  \centering
  % \renewcommand{\arraystretch}{0.8}
  \begin{tabularx}{\textwidth}{ |l|X| }
    \hline
    Name & Log in\\
    \hline
    Actors & User\\
    \hline
    Entry Conditions & \begin{enumerate}[noitemsep]
                        \item The User has opened the application on his device;
                        \item The User has already done the Sign Up activity.
                      \end{enumerate}\\
    \hline
    Event flow & \begin{enumerate}[noitemsep]
                    \item The User chooses the Log-In  option;
                    \item The User fills the email and password fields required for the authentication;
                    \item The User presses the login button.
                \end{enumerate}\\
    \hline
    Exit conditions & The User is logged and the application homepage is displayed on the screen.\\
    \hline
    Exceptions & \begin{enumerate}[noitemsep]
                  \item The User enters the wrong email;
                  \item The User enters the wrong password.
                \end{enumerate}\\
    \hline
  \end{tabularx}
\end{table}
% Send report ---------------
\begin{table}[H]
  \centering
  % \renewcommand{\arraystretch}{0.8}
  \begin{tabularx}{\textwidth}{ |l|X| }
    \hline
    Name & Send report\\
    \hline
    Actors & User\\
    \hline
    Entry Conditions & \begin{enumerate}[noitemsep]
                        \item The User has opened the application on his device;
                        \item The User has already done the Log in activity.
                      \end{enumerate}\\
    \hline
    Event flow & \begin{enumerate}[noitemsep]
                    \item The User presses the button to send a report;
                    \item The User takes a picture of the event;
                    \item The application displays a miniature of the picture and the location information retrieved from the phone's GPS;
                    \item The User chooses the category associated with the type of violation that he wants to report;
                    \item The User optionally adds the plate number of the car and a short description of the violation;
                    \item The User sends the report.
                \end{enumerate}\\
    \hline
    Exit conditions & \begin{enumerate}[noitemsep]
                        \item  The report has been correctly received by the backend system;
                        \item The User can see the status of his report on the homepage of the application.
                      \end{enumerate}\\
    \hline
    Exceptions & \begin{enumerate}[noitemsep]
                  \item The User did not choose the category of violation. The application warns the user to select one type of violation;
                  \item The application misrecognized the location from the GPS. In this case the User can modify the location information.
                \end{enumerate}\\
    \hline
  \end{tabularx}
\end{table}
% Watch the MDS -------------
\begin{table}[H]
  \centering
  % \renewcommand{\arraystretch}{0.8}
  \begin{tabularx}{\textwidth}{ |l|X| }
    \hline
    Name & Watch the MDS\\
    \hline
    Actors & User\\
    \hline
    Entry Conditions & \begin{enumerate}[noitemsep]
                        \item The User has opened the application on his device;
                        \item The User has already done the Log in activity.
                      \end{enumerate}\\
    \hline
    Event flow & \begin{enumerate}[noitemsep]
                    \item The User navigates to the map section;
                    \item The system displays the MDS of the surrounding area centred on the current position of the User;
                    \item If the User chooses a specific address then the MDS will be searched in the area around that address.                
                \end{enumerate}\\
    \hline
    Exit conditions & The User watches the MDS around a specific area. \\ 
    \hline
    Exceptions & The Users inserted an non-existent address. The application warns the User that such address does not exists.
                \\
    \hline
  \end{tabularx}
\end{table}
% Mine information -----------
\begin{table}[H]
  \centering
  % \renewcommand{\arraystretch}{0.8}
  \begin{tabularx}{\textwidth}{ |l|X| }
    \hline
    Name & Mine information\\
    \hline
    Actors & Local system administrator, Police technician\\
    \hline
    Entry Conditions & \begin{enumerate}[noitemsep]
                        \item The Operator has the SafeStreets Web application opened;
                        \item The Operator have already logged in the Web application. 
                      \end{enumerate}\\
    \hline
    Event flow & \begin{enumerate}[noitemsep]
                    \item The Operator selects the option to search information;
                    \item The Operator inserts his desired search criteria;
                    \item The system performs the query based on the inserted filters;
                    \item The system returns the results of the query.              
                \end{enumerate}\\
    \hline
    Exit conditions & The Operator is provided with the desired results.\\ 
    \hline
    Exceptions & The Operator inserted a set of criteria which leads to no results. In that case the Web application warns the Operator with a message that says that no records were found that satisfy the requested query. \\
    \hline
  \end{tabularx}
\end{table}
% Create police technician accounts
\begin{table}[H]
  \centering
  % \renewcommand{\arraystretch}{0.8}
  \begin{tabularx}{\textwidth}{ |l|X| }
    \hline
    Name & Create Police Technician accounts \\
    \hline
    Actors & Local system administrator, Police technician\\
    \hline
    Entry Conditions & \begin{enumerate}[noitemsep]
                        \item The LSA has the SafeStreets Web application opened;
                        \item The LSA have already logged in the Web application;
                        \item The LSA have the mandatory information needed to create a Police technician account.
                      \end{enumerate}\\
    \hline
    Event flow & \begin{enumerate}[noitemsep]
                    \item The LSA selects the "Create Accounts" section in the Web application;
                    \item The LSA insert the mandatory data related to the police officer account that he wants to create (i.e. email, password, badge number);
                    \item The LSA chooses the confirmation option;
                    \item The system saves the new account.
                \end{enumerate}\\
    \hline
    Exit conditions & The new Police technician account has correctly been created.\\ 
    \hline
    Exceptions &  \begin{enumerate}[noitemsep]
                    \item There exist another account with the specified information. The system warns the LSA that an account for that officer already exists;
                    \item The LSA does not fill all the required fields. The Web application warns the user to fill in all the required information.
                  \end{enumerate}\\
    \hline
  \end{tabularx}
\end{table}
% Assign reports to technician 
\begin{table}[H]
  \centering
  % \renewcommand{\arraystretch}{0.8}
  \begin{tabularx}{\textwidth}{ |l|X| }
    \hline
    Name & Assign reports to technician \\
    \hline
    Actors & Local system administrator, Police technician\\
    \hline
    Entry Conditions & \begin{enumerate}[noitemsep]
                        \item The LSA has the SafeStreets Web application opened;
                        \item The LSA have already logged in the Web application;
                        \item The Police technicians have an account registered by the LSA.
                      \end{enumerate}\\
    \hline
    Event flow & \begin{enumerate}[noitemsep]
                    \item The LSA selects the "Schedule" section in the Web application;
                    \item The LSA chooses which violations need one or more Police technician assigned;
                    \item The LSA selects the officers that he wants to assign to the violation;
                    \item The LSA confirms the assignment;
                    \item The system register the schedule.           
                \end{enumerate}\\
    \hline
    Exit conditions & The Police technicians are provided with the violations to which they are associated.\\ 
    \hline
    Exceptions &  \begin{enumerate}[noitemsep]
                    \item The LSA tries to assign officers to a violation with the status "Closed". In that case the application warns the LSA that the violation has already been resolved.
                    \item The LSA tries to associate more than 5 technicians to the same violation. The application warns the LSA that the maximum number of Officers per violation is 5. 
                  \end{enumerate}\\
    \hline
  \end{tabularx}
\end{table}
% TODO: is this good ?
% Consult assigned reports 
\begin{table}[H]
  \centering
  % \renewcommand{\arraystretch}{0.8}
  \begin{tabularx}{\textwidth}{ |l|X| }
    \hline
    Name & Consult assigned reports \\
    \hline
    Actors & Police technician\\
    \hline
    Entry Conditions & \begin{enumerate}[noitemsep]
                        \item The PT has the SafeStreets Web application opened;
                        \item The PT have already logged in the Web application.
                      \end{enumerate}\\
    \hline
    Event flow & \begin{enumerate}[noitemsep]
                    \item The PT selects the "Schedule" section in the Web application;
                    \item The system retrieves the violation reports associated to the PT;
                    \item The Web application displays to the PT the list of violations to which he is assigned;
                    \item The PT can inspect the details of each violation;
                    \item The PT can modify the status of the violation. If he chooses to solve the violation the report will then be archived.           
                \end{enumerate}\\
    \hline
    Exit conditions & The Police technician is provided with the information regarding the reports to which he is assigned.\\ 
    \hline
    Exceptions &  There are no violations assigned to the PT. The Web application then warns the user that there is nothing on his schedule. \\
    \hline
  \end{tabularx}
\end{table}
% Ask for suggestions
\begin{table}[H]
  \centering
  \renewcommand{\arraystretch}{0.6}
  \begin{tabularx}{\textwidth}{ |l|X| }
    \hline
    Name & Ask for suggestions \\
    \hline
    Actors & Local System Administrator, Police technician\\
    \hline
    Entry Conditions & \begin{enumerate}[noitemsep]
                        \item The Operator has the SafeStreets Web application opened;
                        \item The Operator have already logged in the Web application;
                        \item The municipality offers a service to retrieve data related to car accidents.
                      \end{enumerate}\\
    \hline
    Event flow & \begin{enumerate}[noitemsep]
                    \item The Operator selects the "Intervention suggestions" section in the Web application;
                    \item The system asks to the municipality service the data about car accidents happened in the city;
                    \item The system crosses the data received with the violations report submitted in the city;
                    \item The system finds the area where there is a strong correlation between car accidents and violations;
                    \item The system, based on the type of violations, tries to suggest some possible interventions to do on those areas;
                    \item The system displays a list of interventions for each unsafe area.
                    \end{enumerate}\\
    \hline
    Exit conditions & The Operator is provided with a list of possible suggestions.\\ 
    \hline
    Exceptions &  \begin{enumerate}
                    \item The system cannot find a correlation between the information, therefore it is not able to calculate suggestions. In that case the application warns the Operator that no possible interventions have been found;
                    \item The data sent by the municipality are few compared to the violations (or viceversa) so the system cannot make accurate suggestion. In that case the application warns the Operator that not enough data were provided in order to compute possible interventions.
                  \end{enumerate} \\
    \hline
  \end{tabularx}
\end{table}
