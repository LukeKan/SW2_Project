As far as the specification document is concerned, it is necessary to specify some details and to state clearly a few ambiguous points. In order to better clarify those situations, the following assumptions are introduced.
\subsection{Text Assumptions}
\begin{itemize}
    \item \textbf{Violation report}
        \begin{itemize}
            \item The information sent by user in the violation report includes: the license plate image and an optional textual transcription, its position coordinates (extracted automatically from the smartphone GPS) and the violation metadata;
            \item The suitable metadata descripted in the specification document is intended as a choice of the law infringment and a textual description of the event;
            \item Given the coordinates of the violation, the system is able to retrieve the address from which the report was sent;
            \item Every time a report is received, the system will ask the image recognition service for the license plate textual transcription;
            \item A license plate textual transcription is considered correct if it fits into one of the common standards of the EU states. In case of wrong textual transcription of the license plate from the image recognition service:
            \begin{itemize}
                \item The textual transcription provided by the user is taken into account. After a previous check, the provided license plate will be considered as correct;
                \item If the user has not sent a textual transcription, the violation is not associated to any license plate number;
            \end{itemize} 
        \end{itemize}
    \item \textbf{Mining information} 
        \begin{itemize}
            \item End users are allowed to mine information about the violation reports that they sent. Also, they can have access only to the (map visualization or list) MDS;
            \item Authorities can mine information concerning all the stored reports in the SafeStreet system, such as MDS or the list of cars with an high number of violation reports;
        \end{itemize}
    \item \textbf{Suggested intervention}
        \begin{itemize}
            \item Only authorities are allowed to have access to the suggested intervention on unsafe areas identified by the system.
        \end{itemize}
    \end{itemize}
\subsection{Domain assumptions}
\begin{itemize}
    \item {[D1]}: A picture taken with a smartphone is performed with a quality sufficient for the image recognition service to transcribe it;
    \item {[D2]}: The reported picture contains only the license plate of the car that committed the violation and not others;
    \item {[D3]}: The GPS information collected from the smartphone of a user has an accuracy of less than 5 meters;
    \item {[D4]}: The timestamp collected from the smartphone is synchronized with the CET standard;
    \item {[D5]}: The users can only report violations occurred in Europe;
    \item {[D6]}: Reports are only sent through a secure connection channel;
    \item {[D7]}: The users reports all the violation that they detect;
    \item {[D8]}: The users send report containing correct information about the detected violation;
    \item {[D9]}: The municipality system stores correctly all the information concerning car accidents;
    \item {[D10]}: The municipality system allows SafeStreet to retrieve information about the car accident among a certain area;
    \item {[D11]}: The municipality system sends data accordingly to a common standard which is comprehensible by the SafeStreet system;
    \item {[D12]}: The image recognition service is able to detect and transcribe license plates;
    \item {[D13]}: The police corporates and technicians can manage only the violation reports that belong to their jurisdiction;
    \item {[D14]}: The police technicians can only take care of the violations which are assigned to them by the police corporate.
\end{itemize}
