\subsection{Description of the given problem}
SafeStreets is a service that aims to improve the safety of the streets via the help of the users. They can notify violations or any illegal behaviour related to street parking to authorities. In particular users can iteract with the service via an application that can be used to send the violation reports; the latters mainly consist in a picture, taken by the user, of the vehicle responsible for the violation. Moreover users can send, along with the picture, location, date and additional information related to the infringement. The system also provides a Web interface that can be used by authorities in order to check the violations recieved. It is important to note that the picture sent by the user must contain the car plate in order to let police officers know which is the real vehicle that committed the violation. The application also offers the possibility to see areas/streets with the highest violations rates thanks to the data collected over time. As an advanced functionality the application can interact with services offered by the municipality; in particular if a service offers data related to car accidents, SafeStreets can cross this information with its owns, in order to get a better idea of the potentially unsafe areas and therefore suggest some possible interventions. Finally the application will have to be scalable and easy to use in order to provide a fast and efficient utilization for users that see a violation and want to report it immediately.

\subsection{Current system}
SafeStreets is a new service that it is entering the market. There aren't any legacy systems that need to be integrated into the application, with the exception of the third party services that offers some functionality required by the application (like Maps and plate recognition). Those services will be better explained in the following paragraphs.


\subsection{Goals}
% TODO: check if this still hold
\begin{itemize}  
  \item {[G1]}: Allow the user to send a violation report consisting in a picture and some metadata;
  \item {[G2]}: Allow the user to watch the history of his reports and their status;
  \item {[G3]}: Allow the user to watch on map areas and streets with an high frequency of violations;
  \begin{itemize}
    \item {[G3.1]}: The user gets the latest violations near its current location;
    \item {[G3.2]}: The user gets the latest violations near the location that he specifies;
  \end{itemize} 
  \item {[G4]}: Allow the local system administrator of the police station to create accounts for the police technicians;
  \item {[G5]}: Allow the authorities to visualize, schedule and change the status of reports submitted by the users;
  \item {[G6]}: Allow the authorities to mine data to make analysis and retrieve statistics;
  \item {[G7]}: If the municipality offers a service that provides data about car accidents then the system must be able to cross this information with its owns data in order to identify possible unsafe areas;
  \begin{itemize}
    \item {[G.7.1]}: In this case the system must be able to suggest possible interventions;
    \item {[G.7.2]}: The users must be able watch this unsafe areas;
    \item {[G.7.3]}: The authorities must be able to consult the crossed data.
  \end{itemize} 
\end{itemize}

